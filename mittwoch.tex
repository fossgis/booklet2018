\justifying
\newpage


% 10:30
\renewcommand{\conferenceDay}{\mittwoch}
\newsmalltimeslot{10:30}
\abstractZwei{Dominik Helle, Astrid Emde, et.\,al.}%
{Was sind "`Open"' Source, Data und Standards -- und wie funktioniert das?}%
{}%
{Open Source hat viele Facetten -- und es ranken sich inzwischen ebenso viele Mythen darum. Was
davon richtig ist und was nicht, stellen wir in einer kurzen Einführung zusammen. Was sind Open Data
und Open Standards, welche Gemeinsamkeiten gibt es und wo unterscheiden sie sich? Der Vortrag
richtet sich an alle, die mit Open Source, Open Data oder Open Standards bisher noch wenig Kontakt
hatten und die Grundlagen verstehen möchten.}

% 13:00
\newsmalltimeslot{13:00}%
\abstractWPH{Marco Lechner}%
{Eröffnungsveranstaltung der FOSSGIS-Konferenz 2018}%
{}%
{Begrüßung durch den Veranstalter der Konferenz (FOSSGIS e.V.) und Grußworte der gastgebenden Universität}
\enlargethispage{1\baselineskip}

\newsmalltimeslot{13:25}%
\abstractWPH{Andre Caffier}%
{Erfahrungen und Aussichten bei OpenNRW}%
{}%
{}

\newsmalltimeslot{13:50}
\abstractWPH{Marc Vloemans}%
{OSGeo, LocationTech und Open Source als Geschäftsmodell}%
{}{}

\newsmalltimeslot{14:15}%
\abstractWPH{}%
{Lightning Talks}%
{}%
{%
  \vspace{-2em}
  \begin{itemize}
    \RaggedRight
    \setlength{\itemsep}{-2pt} % Aufzählungspunktabstand auf 0
    \item \emph{Jörg Thomsen:} Private Open Data
    \item \emph{Jakob Miksch:} Geodaten mit LightOnEurope intuitiv erkunden
    \item \emph{Till Adams:} Think Big -- der FOSSGIS e.V. als Großkonzern
  \end{itemize}
  \justifying
}

\input{sponsorentexte/201-omniscale.tex}

% 15:00
\newtimeslot{15:00}
\abstractAPH{Markus Neteler}%
{Neues aus dem GRASS GIS Projekt~-- die Version 7.4.0 steht bereit}%
{}%
{
Auch nach über 30 Jahren seit der ersten Version kann GRASS GIS mit modernen Features aufwarten.
Nach fast einjähriger Entwicklungszeit steht die neue stabile Version 7.4 von GRASS GIS zur Verfügung. Das
Entwicklerteam hat die Benutzerfreundlichkeit weiter verbessert, Addons in das Kernpaket migriert
und die Orthorektifikation von Luftbildern überarbeitet. Die Rasterdatenspeicherung wurde auch im
Hinblick auf die Cloud"=Verarbeitung von massiven Datensätzen weiter optimiert und neue
Vektoralgorithmen integriert. Alles steht auch auf dem Docker Hub zur Verfügung.
}

\abstractZwei{Bernhard Ströbl}%
{XPlanung 5.0 in QGIS}%
{}%
{
Die Stadtverwaltung Jena benutzt seit etwa zwei Jahren XPlanung in Verbindung mit QGIS. Sowohl die
Datenmodellierung in PostGIS als auch die Benutzeroberfläche für QGIS wurden komplett im eigenen
Haus entwickelt und stehen als freie Software zur Verfügung. Seit Ende 2017 ist der neue Standard
XPlanung 5.0 implementiert.

Der Vortrag beschreibt die Entwicklung der Datenmodellierung und der
Software sowie ihre Anwendung in der Praxis. Sowohl Entwickler als auch Anwender stehen gern zum
Gespräch und Erfahrungsaustausch zur Verfügung.
}


\abstractVier{Till Adams}%
{Wir reden doch alle Standard -- oder etwa nicht?}%
{Was tun mit WMS, WFS und WCS, SHP, XML und MXD -- MFG?}%
{
GDAL, OGR, PDAL, GEOS und FDO -- und am Ende brauchen doch alle FME?
Die Open-Source-GIS-Welt bietet eine Vielzahl an Lösungen, davon sind viele lebendig, andere
geistern als nicht mehr gepflegte Werkzeuge immer noch im WWW herum.

Der Vortrag stellt Lösungen für Datenkonvertierung und Systemmigration vor dem Schwerpunkt der
Migration von GI"=Werkzeugen anhand von anschaulichen Beispielen vor. Und warnt vor denen, die es
eigentlich nicht mehr gibt.%
}

\newtimeslot{15:35}
% time: 2018-03-21 15:35:00
\abstractAPH{Marc Jansen}%
{OpenLayers}%
{Stand und Zukunft}%
{
OpenLayers ist eine sehr bekannte und verbreitete Open"=Source"=JavaScript"=Bibliothek, um im Web
interaktive Karten aus verschiedensten Quellen mit reichhaltigen Interaktionsmöglichkeiten zu erstellen.
Seit mehr als zehn Jahren wird OpenLayers stets weiterentwickelt und ist auch im Jahre 2018 eine
moderne Bibliothek, die ihren Benutzern eine Fülle an Optionen an die Hand gibt, um auch
anspruchsvollste webbasierte Kartenapplikation zu erstellen.
%
Im Vortrag von OpenLayers-Kernentwicklern werden aktueller Stand und zukünftige Entwicklungen
beleuchtet.%
}

% time: 2018-03-21 15:35:00
\abstractZwei{Dirk Stenger}%
{Datenaustausch in der Bauleitplanung effizienter gestalten mit XPlanung und INSPIRE PLU}%
{}%
{
Dieser Vortrag stellt eine auf Open-Source-Software basierte Lösung vor, um den Datenaustausch in der
Planung effizienter zu gestalten, und zeigt die Abbildung von Prozessen zur Verwaltung der Planwerke
innerhalb einer PostGIS"=Datenhaltung sowie Validierung von XPlanGML"=Dateien. Zudem wird die
Publikation der Daten über deegree Dienste vorgestellt.  Einen Schwerpunkt nimmt die Fragestellung
ein, wie im XPlanGML-Format vorliegende Daten in das INSPIRE Planned Land Use Schema (INSPIRE PLU) überführt
und über INSPIRE-konforme Netzwerkdienste publiziert werden können.%
}

% time: 2018-03-21 15:35:00
\abstractVier{Armin Retterath}%
{WMS Time Dimension}%
{}%
{
Bisher wurden über die Geodateninfrastrukturen des Bundes und der Ländern
grundsätzlich nur Kartendienste mit zeitlich statischen Informationen
publiziert und verwendet. Viele dieser Informationen
unterscheiden sich eigentlich nur durch ihren zeitlichen Bezug (z.\,B. Luftbilder). Die Vielzahl
der entstehenden Datensätze machen insbesondere eine Informationsrecherche über
die Metadaten zunehmend schwieriger. Deshalb steht das
einheitliche Vorgehen bei der Bereitstellung mehrdimensionaler Daten über
Kartendienste auf der Standardisierungsagenda der GDI-DE. Erste Empfehlungen
hierzu sind in dem neuen deutschen Standard für Darstellungsdienste, der dieses Jahr
verabschiedet werden wird, enthalten.

Um einen Einblick in die neuen Möglichkeiten zu eröffnen, wurde im
GeoPortal.rlp der Support für die Zeitdimension implementiert.%
}

\newtimeslot{16:10}
% time: 2018-03-21 16:10:00
\abstractAPH{Pirmin Kalberer}%
{QGIS Web Client 2}%
{}%
{
QGIS Web Client 2 (QWC\,2) ist die zweite Generation des QGIS-Webclients, einem für QGIS Server
optimierten Web-GIS-Client. Er unterstützt die Erweiterungen des QGIS-Servers für den PDF-Druck, die Suche,
den Datenexport, Legenden etc. QWC~2 wurde mit responsivem Design und modular entwickelt. Die identische
Version läuft auf Tablets, Mobiltelefonen und Desktop-Rechnern.

Der Vortrag gibt einen kurzen Überblick über die Funktionalität und zeigt die letzten
Weiterentwicklungen (z.\,B. Digitalisieren). Ebenfalls kurz vorgestellt wird ein exemplarisches
Serversetup.%
}


% time: 2018-03-21 16:10:00
\abstractZwei{David Arndt}%
{Radroutenspeicher Metropole Ruhr}%
{}%
{
Das Geonetzwerk metropoleRuhr ist seit Januar 2016 aktiv in der Abstimmung zur digitalen
Führung der Radwege. Zur Erleichterung der digitalen Erfassung hat der Regionalverband
Ruhr eine Web-GIS Anwendung \emph{Mapbender Radroutenspeicher Metropole Ruhr} sowie ein
QGIS-Projekt erstellt.

Durch die Nutzung der Anwendung und des QGIS"=Projektes wird bei der Pflege
des Datenbestandes "`Radrouten"' sichergestellt, dass keine redundante Datenhaltung bei den
Verbandsmitgliedern vorliegt und gemeinsam in einen Primärdatenbestand eingearbeitet wird.%
}

% time: 2018-03-21 16:10:00
\abstractVier{Just van den Broecke}%
{GeoHealthCheck}%
{Uptime and QoS monitor for geospatial web-services}%
{
  \emph{Dieser Vortrag wird in englischer Sprache gehalten.}

  \begin{otherlanguage}{english}
    Keeping Geospatial Web Services up-and-running is best accommodated by continuous monitoring.  Not
    only downtime needs to be guarded but also whether the services are functioning correctly and do not
    suffer from performance and/or other Quality of Service (QoS) issues.

    GeoHealthCheck is an open source Python application for monitoring OGC Web Services
    uptime and Quality of Service.
  \end{otherlanguage}%
}

\newtimeslot{17:05}
% time: 2018-03-21 17:05:00
\abstractAPH{Nils Bühner}%
{Neuerungen im GeoServer}%
{}%
{
Der GeoServer ist ein mächtiger Open-Source-Kartenserver, der in zahlreichen Projekten eingesetzt
wird. Auf Basis offener Standards können mit dem GeoServer verschiedene Geodienste aus zahlreichen
Datenquellen bereitgestellt werden.

Die GeoServer-Community arbeitet laufend an Erweiterungen und Verbesserungen der Kernsoftware.
Dieser Vortrag widmet sich den Entwicklungen der jüngeren Vergangenheit. Dabei wird das Ziel
verfolgt, einen möglichst breiten Überblick über die (neuen) Möglichkeiten zu schaffen anstatt auf
die Details einzelner Features einzugehen.%
}

% time: 2018-03-21 17:05:00
\abstractZwei{Clemens Rudert}%
{pyramid\_oereb}%
{Kataster öffentlich rechtlicher Eigentumsbeschränkungen auf FOSS-Basis}%
{
Mit \emph{pyramid\_oereb} wurde unter Federführung der Kantone Basel"=Landschaft und Neuenburg sowie
mit der Unterstützung der Firma Camptocamp und der Kantone Schaffhausen, Jura, Basel"=Stadt, Tessin,
Zug und Schwyz ein FOSS-Werkzeugkasten bereitgestellt, welcher die Aufgaben der Datenhaltung,
Zusammenstellung und Darstellung für das ÖREB"=Kataster abdeckt.%
}

% time: 2018-03-21 17:05:00
\abstractVier{Markus Neteler}%
{OpenNRW und Open Source~--\linebreak Verarbeitung von Open (Geo-)Data aus NRW mit Open-Source-Tools}%
{}%
{
Im Vortrag zeigen wir anhand von Fallbeispielen, wie man die neuen, offenen Geodaten von OpenNRW mit
Open-Source-Geo-Werkzeugen verarbeitet. Neben einem Überblick über
vorhandene Daten und verfügbare Werkzeuge werden Beispiele zur Verarbeitung von
LiDAR"=Punktwolkendaten (mit PDAL), eine Landnutzungsklassifizierung aus Orthophotos (mit GRASS GIS
und OTB) sowie die Erstellung von Webservices aus Rasterdaten (mit Geoserver) vorgestellt.%
}

\sponsorenboxA{204-sourcepole.png}{0.4\textwidth}{3}{%
\textbf{Bronzesponsor, Aussteller}\\
Sourcepole bietet Produkte (QGIS Cloud, QGIS Enterprise) und Dienstleistungen
im Bereich Geoinformatik an. Mit mehreren Kernentwicklern von Projekten wie
QGIS oder GDAL kann Sourcepole Herstellersupport übernehmen und
kundenspezifische Erweiterungen und Lösungen umsetzen. www.sourcepole.ch}


\newtimeslot{17:40}
% time: 2018-03-21 17:40:00
\abstractAPH{Sven Böhme}%
{ldproxy -- Geodaten für Jedermann}%
{}%
{
Haben Sie schon einmal Geodaten über OGC-Webdienste genutzt? Mit den richtigen Tools ist das kein Problem,
aber für Entwickler und Nutzer, die sich nur wenig mit Geodaten und deren Standards auskennen, ist
es oft schwierig.

W3C und OGC haben sich mit dem Thema befasst und Empfehlungen dokumentiert, um
die Bereitstellung von Geodaten zu modernisieren und deren Nutzung zu vereinfachen. Wir möchten die
Software \emph{ldproxy} vorstellen, die diese Empfehlungen auf Basis der bestehenden
Geodateninfrastrukturen umsetzt und Hürden bei deren Nutzung reduziert.%
}

% time: 2018-03-21 17:40:00
\abstractZwei{Claas Leiner}%
{ALKIS kompakt mit SpatiaLite}%
{ALKIS-Daten ohne PostGIS-Installation}%
{
Das NAS-XML-Format lässt sich ins QGIS laden, doch müssen die umfänglichen Relationen, die zwischen
Geometrien und Tabellen vorhanden sind, erst wieder aufgebaut werden, um die Daten sinnvoll
nutzen zu können.
Alternativ zu den bewährten Lösungen, die die Installation einer PostGIS"=Datenbank
erfordern, wird hier der Ansatz einer kompakten und transportablen Umsetzung mit SpatiaLite
vorgestellt, bei dem am Ende sämtliche Daten einschließlich Verknüpfungen zwischen Flurstücken und
Ei\-gen\-tü\-mern, sowie einer Auswertung der Flä\-chen\-nut\-zung in einer SQlite-Datei vorliegen.%
}

% time: 2018-03-21 17:40:00
\abstractVier{Sebastian Goerke}%
{OpenMetaData}%
{Metadaten manuell erzeugen war gestern}%
{
Der freie und unbeschränkte Zugang zu Daten ist das zentrale Element des Open-Data-Gedankens. Dieses
Ziel ist durch alleinige Bereitstellung von Daten nicht zu erreichen, denn die bereitgestellten
Daten müssen für interessierte Nutzer auch auffindbar sein. Gerade im Kontext des Bundesministeriums
für Verkehr und Digitale Infrastruktur ist dies eine zentrale Anforderung in Bezug auf die riesigen
Datenschätze des Geschäftsbereiches.

Im Rahmen des Modernitätsfonds mFUND wurde im Projekt OpenMetaData die Machbarkeit von Maßnahmen zur
Verbesserung der Auffindbarkeit von Datensätzen untersucht. Im Fokus steht dabei die Erarbeitung von
auf Open-Source-Software basierenden Lösungsansätzen der Metadatenautomatisierung, die durch Deep"=
Learning"=Verfahren angereichert zur Verbesserung der Datenauffindbarkeit auch hinsichtlich
unterschiedlicher Suchkontexte führen sollen.
}

\newtimeslot{18:15}
% time: 2018-03-21 18:15:00
\abstractAPH{Axel Schaefer}%
{Gute Nachrichten für alle -- eine neue Ausgabe von Mapbender}%
{}%
{
Mapbender ist reich an Möglichkeiten, komplexen wie einfachen
Funktionen. Gleichzeitig versuchen wir nicht nur
Fachanwender abzuholen, sondern auch die Möglichkeit zu bieten, einfach und
schnell Anwendungen zu erzeugen und Informationen zu verknüpfen. Dieser
Spagat ist nicht immer ganz einfach, aber spannend.

In diesem kurzen Vortrag wird ein kleiner Rundumblick mit einzelnen
Beispielen von Neuerungen oder Tipps geliefert.%
}

% time: 2018-03-21 18:15:00
\abstractVier{Martin Over}%
{OpenDEM Europe}%
{Auf dem Weg zu einem freien Geländemodell für Europa}%
{
Im Rahmen der Umsetzung der INPSIRE"=Richtlinie wurden für viele Mitgliedsstaaten der EU
digitale Höhenmodelle unter freien Datenlizenzen veröffentlicht. Die Daten werden mit den
landestypischen horizontalen und vertikalen Referenzsysteme bereit gestellt. Ziel des Projektes ist
eine paneuropäische Nutzung der Daten in einem einheitlichen europaweiten horizontalen und
vertikalen Referenzsystem zu ermöglichen.%
}

% time: 2018-03-21 18:15:00
\abstractZwei{Emmanuel Belo}%
{geOrchestra}%
{INSPIRE GDI mit Fachschalen}%
{
geOrchestra besteht aus einer Open-Source-Community um eine INSPIRE-konforme Geodateninfrastruktur
aufzubauen. Das Ergebnis ist eine anpassbare, interoperable und freie INSPIRE-GDI, basierend auf den
besten Open-Source-Komponenten.%
}

\newsmalltimeslot{18:40}
\label{bof-mittwoch}
\abstractAPH{Axel Schaefer}{Mapbender-Anwendertreffen}{}{}

\newsmalltimeslot{19:15}
\label{social-event}
\RaggedRight
\abstractMensa{}{Campus-Dialoge}{}{%
  \vspace{-1\baselineskip}
  \begin{itemize}
      \setlength{\itemsep}{-0.2\baselineskip}
    \item Buffet
    \item Dialoge
    \item Musik: Dad's Phonkey Loop-Station, DJ-Kollektiv Lucha Amada
  \end{itemize}
}
\justifying
