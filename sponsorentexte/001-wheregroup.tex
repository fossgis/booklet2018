\section*{Goldsponsor}
\begin{center}
	\includegraphics[width=0.6\textwidth]{001_Wheregroup}
\end{center}
Die WhereGroup gehört in Detuschland zu den führenden Anbietern von Geoinformationssystemen mit
Open-Source-Software. Wir bieten alle Dienstleistungen rund um Beratung, Konzeption, Entwicklung,
Aufbau und Betrieb dynamischer Kartenanwendungen im Intra- und Internet. Darüber hinaus gehört ein
umfangreiches Schu\-lungs- und Workshop-Programm zu unserem Portfolio.

Gegründet wurde das Unternehmen als eine Fusion drei verschiedener Unternehmen in Bonn. Im Jahr 2017
haben wir unser 10-jähriges Jubiläum gefeiert. Das WhereGroup"=Team umfasst heute über 30 Angestellte
unterschiedlicher Fachrichtungen -- verteilt auf die Standorte Bonn (Hauptsitz), Freiburg und Berlin.

Das Spektrum unserer Lösungen reicht von Geoportalen und kartenbasierter
Datenverwaltung bis hin zu hochverfügbaren Anwendungen für die freie Wirtschaft und die öffentliche
Verwaltung.

In unseren Projekten setzen wir auf die Standards bzw. Empfehlungen des Open Geospatial
Consortiums (OGC), der INSPIRE"=Richtlinie sowie der GDI"=DE. Ihre Verwendung gewährleistet ein Maximum
an Interoperabilität und Flexibilität unserer Lösungen. Die Einhaltung hoher Sicherheitsstandards
ist für uns nicht zuletzt durch unsere Projekte mit Landes- und Bundesbehörden sowie Großkonzernen
eine Selbst\-ver\-ständ\-lichkeit.

Wir beraten absolut herstellerunabhängig und sind spezialisiert auf die Weiterentwicklung,
professionelle Anwendung und Integration offener Standards und bewährter
Open"=Source"=Technologien und freier Software. Dazu zählen neben unseren Projekten Mapbender\,3,
MetaDor\,2 und PostNAS u.\,a. GeoServer, MapServer, MapProxy, OpenLay\-ers, PostGIS, QGIS und
OpenStreetMap.

Über unser Schulungsinstitut, die FOSS Academy, bieten wir praxisorientierte
Schulungen zum Thema "`GIS mit Open-Source-Software"' an. Diese können sowohl von Einzelpersonen als
auch von Firmen, auf Wunsch auch als Inhouse-Schulungen, gebucht werden.
Zu unserer Überzeugung
gehört, dass wir uns aktiv in die Geoinformatik-Community einbringen. Es ist uns wichtig, an der
Diskussion und Weiterentwicklung von verschiedensten Open-Source-Lösungen mitzuwirken.

\enlargethispage{1.5\baselineskip}
Die WhereGroup ist bundesweit und international mit Hochschulen, Firmen und Verbänden vernetzt. Wir
verfügen über langjährige, persönliche Kontakte zu diversen Universitäten und Hochschulen im In- und
Ausland, zum FOSSGIS e.V., zur Open Source Geospatial Foundation (OSGeo), zum Open Geospatial
Consortium (OGC), sowie zu den Herstellern bzw. Maintainern der gängigsten Open-Source-Produkte im
Geo-Bereich.  Mehr zur WhereGroup unter www.wheregroup.com und www.foss-academy.com.
